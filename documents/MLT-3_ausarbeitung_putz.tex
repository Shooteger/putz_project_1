\documentclass{article}
\usepackage{graphicx}
\graphicspath{ {./images/} }

\title{MLT-3 Simulator}
\date{08-01-2021}
\author{Maurice Putz}

\begin{document}

\maketitle
\pagenumbering{gobble}
\newpage
\pagenumbering{arabic}

\section{Beschreibung}

Der MLT-3 Simulator dient einer Simulierung der Leitungskodierung MLT-3. Vom Sender werde zuerst ASCII-Zeichen 
ins Dezimal System ungewandelt, dann weiter in binären Code und dann mittels der MLT-3 Signalform welche aus
den binären teilen besteht, übertragen. Die MLT-3 codierte Datenfolge wird nun beim Empfänger zurück ins binäre System
übersetzt, danach ins dezimale und schlussendlich wieder als ACII-Zeichen dekodiert ausgegeben.

\subsection{MLT-3 Zusammenfassung}

MLT steht für Multilevel Transmission Encoding, die "3" am Ende steht für 3 levels, also die drei Spannungsformen welche eine
MLT-3 kodierte Datenfolge haben kann. Diese drei Spannungen, was die Folge zu einem Ternären Signal macht,
werden in einer Folge meisten als +, 0 und - Dargestellt, wie man gut in Abb. 1 erkennen kann.



\subsubsection{Subsubsection}

More text.

\paragraph{Paragraph}

Some more text.

\subparagraph{Subparagraph}

Even more text.

\section{Quellen}

\end{document}